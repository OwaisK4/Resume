%%%%%%%%%%%%%%%%%%%%%%%%%%%%%%%%%%%%%%%
% Saif Ul Islam - One Page Two Column Resume
% LaTeX Template
% Version 1.2 (2nd February, 2022)
%
% Original author:
% Debarghya Das (http://debarghyadas.com)
%
% Original repository:
% https://github.com/deedydas/Deedy-Resume
%
% IMPORTANT: THIS TEMPLATE NEEDS TO BE COMPILED WITH XeLaTeX
%
% This template uses several fonts not included with Windows/Linux by
% default. If you get compilation errors saying a font is missing, find the line
% on which the font is used and either change it to a font included with your
% operating system or comment the line out to use the default font.
% 
%%%%%%%%%%%%%%%%%%%%%%%%%%%%%%%%%%%%%%
% 
% TODO:
% 1. Integrate biber/bibtex for article citation under publications.
% 2. Figure out a smoother way for the document to flow onto the next page.
% 3. Add styling information for a "Projects/Hacks" section.
% 4. Add location/address information
% 5. Merge OpenFont and MacFonts as a single sty with options.
% 
%%%%%%%%%%%%%%%%%%%%%%%%%%%%%%%%%%%%%%
%
% CHANGELOG:
% v1.1:
% 1. Fixed several compilation bugs with \renewcommand
% 2. Got Open-source fonts (Windows/Linux support)
% 3. Added Last Updated
% 4. Move Title styling into .sty
% 5. Commented .sty file.
%
%%%%%%%%%%%%%%%%%%%%%%%%%%%%%%%%%%%%%%%
%
% Known Issues:
% 1. Overflows onto second page if any column's contents are more than the
% vertical limit
% 2. Hacky space on the first bullet point on the second column.
%
%%%%%%%%%%%%%%%%%%%%%%%%%%%%%%%%%%%%%%


\documentclass[]{openfont}
\usepackage{fancyhdr}
 
\pagestyle{fancy}
\fancyhf{}
 
\begin{document}

%%%%%%%%%%%%%%%%%%%%%%%%%%%%%%%%%%%%%%
%
%     LAST UPDATED DATE
%
%%%%%%%%%%%%%%%%%%%%%%%%%%%%%%%%%%%%%%
\lastupdated

%%%%%%%%%%%%%%%%%%%%%%%%%%%%%%%%%%%%%%
%
%     TITLE NAME
%
%%%%%%%%%%%%%%%%%%%%%%%%%%%%%%%%%%%%%%
\namesection{}{Owais Ali Khan}{ \href{https://www.linkedin.com/in/owais-ali-khan-04933b238}{linkedin/in/owais-ali-khan} | \href{https://github.com/OwaisK4}{github.com/OwaisK4} \\
    \href{mailto:owaisalikhan2003@gmail.com}{owaisalikhan2003@gmail.com} | +92 322 392 1437 \\
    \href{https://codeforces.com/profile/snowden\_03}{codeforces.com/profile/snowden\_03}
}

%%%%%%%%%%%%%%%%%%%%%%%%%%%%%%%%%%%%%%
%
%     COLUMN ONE
%
%%%%%%%%%%%%%%%%%%%%%%%%%%%%%%%%%%%%%%

\begin{minipage}[t]{0.33\textwidth}

    %%%%%%%%%%%%%%%%%%%%%%%%%%%%%%%%%%%%%%
    %     EDUCATION
    %%%%%%%%%%%%%%%%%%%%%%%%%%%%%%%%%%%%%%

    \section{Education}

    \subsection{FAST NUCES, Karachi}
    \descript{BS(CS), 3.89 CGPA}
    \location{July 2025}

    %%%%%%%%%%%%%%%%%%%%%%%%%%%%%%%%%%%%%%
    %     ACHIEVEMENTS
    %%%%%%%%%%%%%%%%%%%%%%%%%%%%%%%%%%%%%%

    \section{Achievements}
    \textbullet{} \location{Gold Medallist ICPC Asia Topi Regional '24-25 {\footnotesize \textit{\textbf{(National 4th rank)}}}}

    \textbullet{} \location{Silver Medallist ICPC Asia Topi Regional '23-24 {\footnotesize \textit{\textbf{(National 8th rank)}}}}

    \textbullet{} \location{Ranked 531/4200, IEEE Xtreme Programming Competition 17.0 {\footnotesize \textit{\textbf{(Global 531st rank)}}}}

    \textbullet{} \location{Rector's List {\footnotesize \textit{\textbf{(Spring '22)}}}}

    \textbullet{} \location{Dean's List {\footnotesize \textit{\textbf{(Fall '21, Fall '22, Spring '23, Fall '23, Spring '24, Fall '24)}}}}

    %%%%%%%%%%%%%%%%%%%%%%%%%%%%%%%%%%%%%
    %     CERTIFICATIONS
    %%%%%%%%%%%%%%%%%%%%%%%%%%%%%%%%%%%%%%

    % \section{Certifications}

    % \textbullet{} SUSE Scholarship Challenge Finalist 2021 @Udacity, @SUSE, \location{\href{https://udacity-email.s3.us-west-2.amazonaws.com/SUSE/SUSE_Scholarship_Finalist_Badge.png?bsft_aaid=affd8710-61ff-4001-baca-1d4a7303381d&bsft_eid=c10c73ad-a3fc-4be5-8a8e-406fe76a2062&utm_campaign=sch_600_2021-08-04_ndxxx_suse-100-completion-badge&utm_source=blueshift&utm_medium=email&utm_content=sch_600_2021-08-04_ndxxx_suse-100-completion-badge&bsft_clkid=98dac48e-82fb-4a17-9422-990158e04d87&bsft_uid=1ad320c9-3e03-4e9c-bf57-15c259eb60cb&bsft_mid=e9b16371-15d0-4f21-800d-beb044dc1d74&bsft_mime_type=html&bsft_ek=2021-08-04T13\%3A22\%3A39Z&bsft_lx=4&bsft_tv=7}{(2021)}}

    % \textbullet{} AWS Fundamentals @ Udacity, \location{\href{https://user-images.githubusercontent.com/41635766/128562528-47a5799b-f362-4df6-ab6a-42a52d230415.png}{(2020)}}

    % \textbullet{} Google Cloud Fundamentals: Core Infrastructure, \location{\href{https://s3.amazonaws.com/coursera_assets/meta_images/generated/CERTIFICATE_LANDING_PAGE/CERTIFICATE_LANDING_PAGE~BKZKFUZPJDW5/CERTIFICATE_LANDING_PAGE~BKZKFUZPJDW5.jpeg}{(2020)}}

    % \textbullet{} Google Cloud, Big Data \& Machine Learning Fundamentals, \location{\href{https://s3.amazonaws.com/coursera_assets/meta_images/generated/CERTIFICATE_LANDING_PAGE/CERTIFICATE_LANDING_PAGE~5PH3QTYTHEMM/CERTIFICATE_LANDING_PAGE~5PH3QTYTHEMM.jpeg}{(2020)}}

    % \textbullet{} Continuous Delivery \& DevOps - University Of Virginia, Darden School Of Business, \location{\href{https://s3.amazonaws.com/coursera_assets/meta_images/generated/CERTIFICATE_LANDING_PAGE/CERTIFICATE_LANDING_PAGE~836VWQQJ9F57/CERTIFICATE_LANDING_PAGE~836VWQQJ9F57.jpeg)}{(2020)}}

    % \textbullet{} Deep Learning Specialization @DeepLearning.AI, \location{\href{https://s3.amazonaws.com/coursera_assets/meta_images/generated/CERTIFICATE_LANDING_PAGE/CERTIFICATE_LANDING_PAGE~TXFEKNJQ7WVE/CERTIFICATE_LANDING_PAGE~TXFEKNJQ7WVE.jpeg}{(2020)}}

    % \textbullet{} DeepLearning.AI Tensorflow Developer, \location{\href{https://s3.amazonaws.com/coursera_assets/meta_images/generated/CERTIFICATE_LANDING_PAGE/CERTIFICATE_LANDING_PAGE~57GG4X4J48HY/CERTIFICATE_LANDING_PAGE~57GG4X4J48HY.jpeg}{(2020)}}

    % \textbullet{} Machine Learning @ Stanford, \location{\href{https://s3.amazonaws.com/coursera_assets/meta_images/generated/CERTIFICATE_LANDING_PAGE/CERTIFICATE_LANDING_PAGE~6XQ5BSTL6MSG/CERTIFICATE_LANDING_PAGE~6XQ5BSTL6MSG.jpeg}{(2020)}}

    %%%%%%%%%%%%%%%%%%%%%%%%%%%%%%%%%%%%%
    %     SKILLS
    %%%%%%%%%%%%%%%%%%%%%%%%%%%%%%%%%%%%%%

    \section{Skills}

    \subsection{Languages}
    Python \textbullet{} C/C++ \textbullet{} Java \textbullet{} JavaScript/TypeScript \textbullet{} Dart
    \sectionsep

    \subsection{Libraries/Frameworks}
    Django \textbullet{} Flutter \textbullet{} NodeJS \textbullet{} ExpressJS \textbullet{} Flask \textbullet{} Bootstrap5 \textbullet{} Material UI \textbullet{} CMake \textbullet{} CUDA
    \sectionsep

    \subsection{Productivity}
    Git \textbullet{} Linux (Debian-based) \textbullet{} Zsh/Bash (Shell Scripting) \textbullet{} Vim \textbullet{} Markdown \textbullet{} LaTeX \textbullet{} Figma
    \sectionsep

    \subsection{Databases}
    MySQL \textbullet{} PostGRESql
    \sectionsep

    \subsection{DevOps / Cloud}
    Docker \textbullet{} Jenkins (CI/CD)
    % Docker \textbullet{} Kubernetes \textbullet{} Jenkins (CI/CD)
    \sectionsep

    %%%%%%%%%%%%%%%%%%%%%%%%%%%%%%%%%%%%%%
    %
    %     COLUMN TWO
    %
    %%%%%%%%%%%%%%%%%%%%%%%%%%%%%%%%%%%%%%

\end{minipage}
\hfill
\begin{minipage}[t]{0.65\textwidth}

    %%%%%%%%%%%%%%%%%%%%%%%%%%%%%%%%%%%%%%
    %     EXPERIENCE
    %%%%%%%%%%%%%%%%%%%%%%%%%%%%%%%%%%%%%%

    \section{Experience}
    \runsubsection{\href{https://securiti.ai/}{Securiti AI}}
    \descript{| Software Engineer Intern}
    \location{June '24 - Aug '24 | Hybrid}
    \vspace{\topsep}
    \begin{tightemize}
        \item Worked in the \textit{Content Classification} (CC) \textit{team}.
        \item Built an infrastructure for matching large scale \textit{regular expressions} using state-of-the-art open source tools (i.e \href{https://www.intel.com/content/www/us/en/developer/articles/technical/introduction-to-hyperscan.html}{\textbf{Intel Hyperscan}}), and wrote integrations with the existing \textit{entity extraction} workflow.
    \end{tightemize}
    \sectionsep

    \runsubsection{\href{https://retrocausal.ai/}{Retrocausal AI}}
    \descript{| Applied Research Intern}
    \location{Mar '24 - June '24 | Remote}
    % \vspace{\topsep}
    \begin{tightemize}
        \item Fulfilled my duties as part of the \textit{Applied Research} team.
        \item Created workflows and automations for \textit{reporting} and \textit{exporting} internal data models in standard formats.
    \end{tightemize}
    \sectionsep

    \runsubsection{\href{https://ldplogistic.com/}{LDP Logistics}}
    \descript{| IT Support Specialist}
    \location{June '23 - Sep '23 | Onsite}
    % \vspace{\topsep}
    \begin{tightemize}
        \item Backend \textit{developer} and sole \textit{maintainer} for the internal management website used in local and overseas offices.
        \item In addition, the main IT \textit{solutions provider} and technical officer for all in-house teams.
    \end{tightemize}
    \sectionsep

    %%%%%%%%%%%%%%%%%%%%%%%%%%%%%%%%%%%%%%
    %     PROJECTS
    %%%%%%%%%%%%%%%%%%%%%%%%%%%%%%%%%%%%%%

    \section{Projects}
    \runsubsection{\href{https://github.com/OwaisK4/Parallel_Image_Enhancement}{Polaris}}
    \descript{| High-performance Image processing application}
    \location{Fall 2023 | Parallel and Distributed Computing (Project)}
    \begin{tightemize}
        \item Designed and developed \textbf{\href{https://github.com/OwaisK4/Parallel_Image_Enhancement}{Polaris}}, a \textit{parallelized} image enhancement program. Polaris performs standard image enhancement techniques (i.e. \href{https://en.wikipedia.org/wiki/Gaussian_filter}{\textbf{Gaussian filtering}} and \href{https://en.wikipedia.org/wiki/Histogram_equalization}{\textbf{Histogram Equalization}}) in order to \textit{sharpen} image quality and increase \textit{contrast} in dark images.\\

        \item Polaris is written in \textbf{\href{https://en.cppreference.com/w/cpp/17}{C++17}} and uses the open-source \textit{multiprogramming} framework \textbf{\href{https://www.openmp.org/}{OpenMP}} to utilize all available \textit{CPU cores} for image processing. \\
    \end{tightemize}
    \sectionsep

    \runsubsection{\href{https://github.com/OwaisK4/Inventory_Manager_Django}{LDP Asset Manager}}
    \descript{| Inventory Management System}
    \location{Summer 2023 | Full Stack Development}
    \begin{tightemize}
        \item Developed an \textit{inventory management system} using \textbf{Django}, \textbf{MUI}, \textbf{JQuery}, \textbf{MySQL}, \textbf{Apache HTTP Server}, \textbf{Bootstrap4}
        \item The main function of this project is to act as a \textit{web portal} that keeps track of all the inventory (assets, accessories, licenses, etc) held at all \textit{LDP} offices.
        \item Deployed the finished web application on \textbf{\href{https://www.dreamhost.com/}{Dreamhost}}
    \end{tightemize}

    \runsubsection{\href{https://github.com/OwaisK4/Video_Classification}{Tagger}}
    \descript{| Data Extraction and Analysis}
    \location{Spring 2024 | Information Retrieval (Project)}
    \begin{tightemize}
        \item Created a tool that uses \textit{Azure} services (\textbf{SpeechService} and \textbf{TextAnalytics}) to \textit{transcribe} video and extract \textit{NamedEntities} (i.e. landmarks).
        \item Also created a \textit{search} mechanism for the recognized entities from an existing \textit{knowledgebase}  (i.e. Tripadvisor).
        % \item The first part takes a video as an input, transcribes it using Azure SpeechService and outputs a transcript of the video.
        % \item The second part takes the the transcript and performs NamedEntityRecognition (NER) tagging to get relevant keywords (i.e. landmarks).
        % \item The final part takes those keywords and queries a knowledgebase (i.e. Tripadvisor) for information pertaining to that keyword.
        % \textbf{\href{https://www.dreamhost.com/}{Dreamhost}}
    \end{tightemize}

    %%%%%%%%%%%%%%%%%%%%%%%%%%%%%%%%%%%%%
    %     VOLUNTEERING & COMMUNITIES
    %%%%%%%%%%%%%%%%%%%%%%%%%%%%%%%%%%%%%%

    \section{Volunteering \& Communities}
    \textbullet{} \location{IEEE Membership {\footnotesize{\textit{\textbf{(Dec '23 - Dec '24)}}}}}

    \textbullet{} \location{Procom '23 Co-head CS Competitions {\footnotesize{\textit{\textbf{(Jan '23 - Mar '23)}}}}}

    \textbullet{} \location{Procom '22 Co-head Math Olympiad {\footnotesize{\textit{\textbf{(Jan '22 - Mar '22)}}}}}

    \sectionsep

\end{minipage}
\end{document}  \documentclass[]{article}
